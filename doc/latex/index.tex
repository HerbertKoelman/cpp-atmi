\subsubsection*{What it does}

A\+T\+M\+I, for Application-\/to-\/\+Transaction Monitor Interface, is the main A\+P\+I for the Tuxedo system. It includes transaction management functions (routines, verbs); request/response, conversational, queuing, and publish-\/and-\/subscribe message-\/handling functions; service interface functions; and buffer management functions for distributed application communication.

A\+T\+M\+I++ is a C++ wrapping of this A\+P\+I, adding strong typing of F\+M\+L buffers and adding the use of exceptions to detect and handle error conditions.

This should help making it easier to write fast and rock solid Tuxedo software.

This project is currently hosted \href{http://herbertkoelman.github.com/cpp-atmi}{\tt here}.

\subsubsection*{Setting things up}

\begin{DoxyVerb}$ configure
$ make
\end{DoxyVerb}


Documentation is obtained with this command\+: \begin{DoxyVerb}$ make doxygen
\end{DoxyVerb}


The library is documented \href{http://herbertkoelman.github.io/cpp-atmi/doc/html/}{\tt here}.

\subsubsection*{How it's used}

This code creates a F\+M\+L field and passes it to a service called {\ttfamily X\+A\+T\+O\+U\+P\+P\+E\+R}. If anything goes wrong, we roll back the transaction and pass the exception ({\ttfamily throw}) so that the diagnostic messages can be displayed. \begin{DoxyVerb}    ...
    transaction tp ( "XATOUPPER" );
    int urcode = 0;

    try {

      buffer buffer;

      Tfield<string> name ( "EMPNAME" );
      name = (string) "Herbert.Koelman@me.com";

      buffer.add ( &name );

      printf ("Calling service TOUPPER (%s), context: %d.\n", name.c_str() , name());

      try {

        tp.begin();

        int ret = tp.call ( &buffer );
        switch (ret) {
          case 0:
            break;
          case -1:
            printf ("TOUPPER returned TPFAIL.\n" );
            break;
          default:
            printf ( "TOUPPER returned: %d (error: %d/%d).\n", ret, tp.error());
            buffer.print();
        }

        buffer.get ( &name );
        printf ( "Call returned: %s.\n", name.c_str() );

        tp.commit();
      } catch ( ... ){
        tp.abort();
        throw;
      }

    } catch ( tuxedo_exception& err ) {
      printf ( "Tuxedo Exception catched. tpcall failed. %s\n", err.what() ); 
    };
    ...
\end{DoxyVerb}


\subsubsection*{License}





A\+T\+M\+I++ is free software, you can redistribute it and/or modify it under the terms of the G\+N\+U General Public License as published by the Free Software Foundation; either version 2 of the License, or (at your option) any later version.

A\+T\+M\+I++ is distributed in the hope that it will be useful, but W\+I\+T\+H\+O\+U\+T A\+N\+Y W\+A\+R\+R\+A\+N\+T\+Y; without even the implied warranty of M\+E\+R\+C\+H\+A\+N\+T\+A\+B\+I\+L\+I\+T\+Y or F\+I\+T\+N\+E\+S\+S F\+O\+R A P\+A\+R\+T\+I\+C\+U\+L\+A\+R P\+U\+R\+P\+O\+S\+E. See the G\+N\+U General Public License for more details.

You should have received a copy of the G\+N\+U General Public License along with A\+T\+M\+I++; if not, write to the Free Software Foundation, Inc., 51 Franklin St, Fifth Floor, Boston, M\+A 02110-\/1301 U\+S\+A 

 Copyright (C) 2016 -\/ herbert koelman 